\documentclass[5pt,]{article}
\usepackage[T1]{fontenc}
\usepackage{lmodern}
\usepackage{amssymb,amsmath}
\usepackage{ifxetex,ifluatex}
\usepackage{fixltx2e} % provides \textsubscript
% use upquote if available, for straight quotes in verbatim environments
\IfFileExists{upquote.sty}{\usepackage{upquote}}{}
\ifnum 0\ifxetex 1\fi\ifluatex 1\fi=0 % if pdftex
  \usepackage[utf8]{inputenc}
\else % if luatex or xelatex
  \ifxetex
    \usepackage{mathspec}
    \usepackage{xltxtra,xunicode}
  \else
    \usepackage{fontspec}
  \fi
  \defaultfontfeatures{Mapping=tex-text,Scale=MatchLowercase}
  \newcommand{\euro}{€}
\fi
% use microtype if available
\IfFileExists{microtype.sty}{\usepackage{microtype}}{}
\usepackage[margin=0.45in]{geometry}
\usepackage{graphicx}
% Redefine \includegraphics so that, unless explicit options are
% given, the image width will not exceed the width of the page.
% Images get their normal width if they fit onto the page, but
% are scaled down if they would overflow the margins.
\makeatletter
\def\ScaleIfNeeded{%
  \ifdim\Gin@nat@width>\linewidth
    \linewidth
  \else
    \Gin@nat@width
  \fi
}
\makeatother
\let\Oldincludegraphics\includegraphics
{%
 \catcode`\@=11\relax%
 \gdef\includegraphics{\@ifnextchar[{\Oldincludegraphics}{\Oldincludegraphics[width=\ScaleIfNeeded]}}%
}%
\ifxetex
  \usepackage[setpagesize=false, % page size defined by xetex
              unicode=false, % unicode breaks when used with xetex
              xetex]{hyperref}
\else
  \usepackage[unicode=true]{hyperref}
\fi
\hypersetup{breaklinks=true,
            bookmarks=true,
            pdfauthor={},
            pdftitle={The effect of the transmission on fuel consumption},
            colorlinks=true,
            citecolor=blue,
            urlcolor=blue,
            linkcolor=magenta,
            pdfborder={0 0 0}}
\urlstyle{same}  % don't use monospace font for urls
\setlength{\parindent}{0pt}
\setlength{\parskip}{6pt plus 2pt minus 1pt}
\setlength{\emergencystretch}{3em}  % prevent overfull lines
\setcounter{secnumdepth}{0}

\title{The effect of the transmission on fuel consumption}
\author{}
\date{}

\begin{document}

\begin{center}
\huge The effect of the transmission on fuel consumption \\[0.2cm]
\normalsize
\end{center}


\subsubsection{Executive summary}\label{executive-summary}

This report uses data extracted from the 1974 Motor Trend US magazine
that comprises fuel consumption and 10 aspects of automobile design and
performance for 32 automobiles (1973-74 models). This data is used to
estimate the effect of the type of transmission (automatic vs.~manual)
on the milage (MPG). The results suggest that manual transmission allows
for a higher MPG. However this result only becomes statistically
significant after a model adjustment that attributes more explanatory
power to the transmission.

\subsubsection{Introduction and exploratory data
analysis}\label{introduction-and-exploratory-data-analysis}

The starting point for this analysis is a simple theory based approach.
Based on intuition it seems reasonalbe to assume that the mpg of a car
is mainly determined by its weight and its power. Furthermore it is
likely that cars with more cylinders have a lager displacement and
therefore also more hp. The following short exploratory data analysis
verifies these assumptions.
\includegraphics{./regCP_files/figure-latex/unnamed-chunk-2.pdf}

The left part of above figure plots the MPG against the weight using
colors to show the number of cylinders an symbols to represent the
transmission. The plot provides a couple of interesting insights: (1.1)
There is a clear negative relationship between MPG and weight, (1.2)
there is a clear positive relationship between the number of cylinders
and the weight of a car and (1.3) automatic transmission seems to be
used almost exclusively in heavy cars with 8 cylinder engines whereas
manual transmissions are mainly used in lighter cars with 4 cylinder
engines. The right part shows that (2.1) there is a strong linear
positive relationship between HP and displacement (2.2) engines with
more cylinders clearly also provide more HP and (2.3) apart from two
excpetions all high-powered cars use an automatic transmission.

\subsubsection{The initial model}\label{the-initial-model}

The exploratory data analysis prooves that a model using the weight and
some variable measuring the power of a car is a reasonable starting
point. However (2.1) and (2.2) suggest that a model containing all of
the three power-relatded variables (\texttt{cyl, disp, hp}) will suffer
from multicollinearity. Therefore the initial model will only use cyl as
a factor-varible. In addition (1.3) suggests that including weight in
the model will suppress any potential impact of the transmission on
\texttt{mpg}. The initial model for this analysis is therefore
\texttt{mpg \textasciitilde{} am + cyl + wt} (whereas \texttt{am} and
\texttt{cyl} a treated as factor variables). The estimation results for
this regression can be found in the appendix and show that the
coefficients on both \texttt{wt} and \texttt{cyl} are statistically
significant on a 1\% significance level both coefficients feature
expected signs, i.e.~higher \texttt{wt} implies lower \texttt{mpg} and
more cylinders imply lower \texttt{mpg} as well. Furthermore we cannot
reject the hypothesis that the coefficient on \texttt{am} is 0.

\subsubsection{Strategy for model enhancement and
selection}\label{strategy-for-model-enhancement-and-selection}

From the exploratory data analysis we have seen that automatic
transmissions are mainly used in heavy, high-powered cars. We cannot say
anything about the causal direction of the correlation but theory
suggests that an automatic transmission is heavier than a manual
transmission which will result in a lower \texttt{mpg}. However, in the
initial model this lower \texttt{mpg} is attributed to the variable
\texttt{wt} and not to \texttt{am}. Therefore my approach is to adjust
\texttt{wt} to attribute any weight-related effects of the transmission
to \texttt{am}. This is done by an auxialliary regression where I
regress \texttt{wt} on \texttt{am} and some other explanatory variables
from the \texttt{mtcars} dataset. I used likelihood ratio tests to
determine the correct model for this auxilliary regression. The
estimation of the auxilliary regression shows that, when controlling for
other factors, cars with automatic transmissions tend to be 700lbs
heavier than cars with manual transmissions. Details on the results
including residual plots and diagnostics can be found in the appendix.
The residual plot and the diagnostics suggest that the model is valid.

To retrieve the adjusted model I subtract these 700lbs from all cars
with a automatic transmission and store the new weight in a variable
\texttt{mwt}:
\texttt{mtcars\$mwt \textless{}- mtcars\$wt + (1-mtcars\$am)*-0.70885}.

The adjusted base model is \texttt{mpg\textasciitilde{}am+cyl+mwt}.
Starting from this base model I gradually add the variables \texttt{hp},
\texttt{drat} and \texttt{vs} select the final model based on a
likelihood ratio test. The likelihood ratio test suggests that
\texttt{hp} should be added to the model. The detailed results can be
found in the appendix.

\subsubsection{Estimating the final model and interpreting the
results}\label{estimating-the-final-model-and-interpreting-the-results}

The final model used to estimate the impact of the transmission type on
the mileage of the car is therefore
\texttt{mpg\textasciitilde{}am+cyl+mwt+hp}. The estimastion results are
given below.

\begin{verbatim}

Call:
lm(formula = mpg ~ factor(am) + factor(cyl) + mwt + hp, data = mtcars)

Residuals:
   Min     1Q Median     3Q    Max 
-3.939 -1.256 -0.401  1.125  5.051 

Coefficients:
             Estimate Std. Error t value Pr(>|t|)    
(Intercept)   31.9384     2.0794   15.36  1.5e-14 ***
factor(am)1    3.5791     1.1392    3.14   0.0042 ** 
factor(cyl)6  -3.0313     1.4073   -2.15   0.0407 *  
factor(cyl)8  -2.1637     2.2843   -0.95   0.3523    
mwt           -2.4968     0.8856   -2.82   0.0091 ** 
hp            -0.0321     0.0137   -2.35   0.0269 *  
---
Signif. codes:  0 '***' 0.001 '**' 0.01 '*' 0.05 '.' 0.1 ' ' 1

Residual standard error: 2.41 on 26 degrees of freedom
Multiple R-squared:  0.866, Adjusted R-squared:  0.84 
F-statistic: 33.6 on 5 and 26 DF,  p-value: 1.51e-10
\end{verbatim}

The estimation results from my final model report a positive coefficient
on \texttt{am}, i.e.~a manual transmission is better for MPG. Recall
that \texttt{am} is 0 for cars with an automatic transmission and 1 for
cars with a manual transmission. The p-value for \texttt{am} is
\texttt{0.004}, therefore the coefficient is considered statistically
significant on a 1\% significance level. Additionally the estimation
results support the hypothesis that a higher weight, more hp and also
more cylinders yield a lower MPG.

The value of the coefficient on \texttt{am} is \texttt{3.58}. Therefore,
holding the other factors constant, the MPG of a car with a manual
transmission is 3.58 higher than the MPG of a car with an automatic
transmission. To quantify the uncertainty in this conclusion I compute
the 95\% confidence interval. The 95\% confidence interval for
\texttt{am} is {[}1.2373, 5.9208{]}. Therefore we can say that we are
95\% confident that the true effect of \texttt{am} is in this range.
Furthermore the R\^{}2 shows that the model explains 86,6\% of the
variation in
MPG.\\\includegraphics{./regCP_files/figure-latex/unnamed-chunk-4.pdf}

The residual plot shows no systematic trend in the residuals.
Furtheremore the Q-Q plot does not indicate a non-normal distribution of
the residuals. The appendix shows additional diagnostics for the
estimation. Looking at the Cook's distance and the residual vs leverage
plot in the appendix one can tell none of the single observations has a
critical influence on the model. All in all the diagnostics give
indication that the chosen model is valid.

\subsection{Appendix}\label{appendix}

\subsubsection{Correlation matrix}\label{correlation-matrix}

\begin{verbatim}
       mpg   cyl  disp    hp  drat    wt  qsec    vs    am  gear  carb
mpg   1.00 -0.85 -0.85 -0.78  0.68 -0.87  0.42  0.66  0.60  0.48 -0.55
cyl  -0.85  1.00  0.90  0.83 -0.70  0.78 -0.59 -0.81 -0.52 -0.49  0.53
disp -0.85  0.90  1.00  0.79 -0.71  0.89 -0.43 -0.71 -0.59 -0.56  0.39
hp   -0.78  0.83  0.79  1.00 -0.45  0.66 -0.71 -0.72 -0.24 -0.13  0.75
drat  0.68 -0.70 -0.71 -0.45  1.00 -0.71  0.09  0.44  0.71  0.70 -0.09
wt   -0.87  0.78  0.89  0.66 -0.71  1.00 -0.17 -0.55 -0.69 -0.58  0.43
qsec  0.42 -0.59 -0.43 -0.71  0.09 -0.17  1.00  0.74 -0.23 -0.21 -0.66
vs    0.66 -0.81 -0.71 -0.72  0.44 -0.55  0.74  1.00  0.17  0.21 -0.57
am    0.60 -0.52 -0.59 -0.24  0.71 -0.69 -0.23  0.17  1.00  0.79  0.06
gear  0.48 -0.49 -0.56 -0.13  0.70 -0.58 -0.21  0.21  0.79  1.00  0.27
carb -0.55  0.53  0.39  0.75 -0.09  0.43 -0.66 -0.57  0.06  0.27  1.00
\end{verbatim}

\subsubsection{Estimation of the initial
model}\label{estimation-of-the-initial-model}

\begin{verbatim}

Call:
lm(formula = mpg ~ factor(am) + factor(cyl) + wt, data = mtcars)

Residuals:
   Min     1Q Median     3Q    Max 
-4.490 -1.312 -0.504  1.416  5.776 

Coefficients:
             Estimate Std. Error t value Pr(>|t|)    
(Intercept)    33.754      2.813   12.00  2.5e-12 ***
factor(am)1     0.150      1.300    0.12   0.9089    
factor(cyl)6   -4.257      1.411   -3.02   0.0055 ** 
factor(cyl)8   -6.079      1.684   -3.61   0.0012 ** 
wt             -3.150      0.908   -3.47   0.0018 ** 
---
Signif. codes:  0 '***' 0.001 '**' 0.01 '*' 0.05 '.' 0.1 ' ' 1

Residual standard error: 2.6 on 27 degrees of freedom
Multiple R-squared:  0.838, Adjusted R-squared:  0.813 
F-statistic: 34.8 on 4 and 27 DF,  p-value: 2.73e-10
\end{verbatim}

\subsubsection{Auxilliary regression to estimate impact of transmission
on
weight}\label{auxilliary-regression-to-estimate-impact-of-transmission-on-weight}

\begin{verbatim}
Analysis of Variance Table

Model 1: wt ~ factor(cyl) + disp + factor(am)
Model 2: wt ~ factor(cyl) + disp + factor(am) + carb
Model 3: wt ~ factor(cyl) + disp + factor(am) + carb + hp
  Res.Df  RSS Df Sum of Sq    F Pr(>F)   
1     27 4.51                            
2     26 3.41  1     1.093 8.96 0.0061 **
3     25 3.05  1     0.363 2.98 0.0968 . 
---
Signif. codes:  0 '***' 0.001 '**' 0.01 '*' 0.05 '.' 0.1 ' ' 1
\end{verbatim}

\begin{verbatim}

Call:
lm(formula = wt ~ factor(cyl) + disp + factor(am) + carb, data = mtcars)

Residuals:
    Min      1Q  Median      3Q     Max 
-0.6616 -0.2956  0.0158  0.2463  0.6095 

Coefficients:
             Estimate Std. Error t value Pr(>|t|)    
(Intercept)   1.79024    0.24502    7.31  9.3e-08 ***
factor(cyl)6 -0.25131    0.23949   -1.05  0.30367    
factor(cyl)8 -0.81067    0.38429   -2.11  0.04468 *  
disp          0.00723    0.00139    5.19  2.1e-05 ***
factor(am)1  -0.70885    0.18478   -3.84  0.00072 ***
carb          0.16243    0.05630    2.89  0.00776 ** 
---
Signif. codes:  0 '***' 0.001 '**' 0.01 '*' 0.05 '.' 0.1 ' ' 1

Residual standard error: 0.362 on 26 degrees of freedom
Multiple R-squared:  0.885, Adjusted R-squared:  0.863 
F-statistic:   40 on 5 and 26 DF,  p-value: 2.11e-11
\end{verbatim}

\includegraphics{./regCP_files/figure-latex/unnamed-chunk-8.pdf}

\subsubsection{Steps to the final model}\label{steps-to-the-final-model}

\begin{verbatim}
Analysis of Variance Table

Model 1: mpg ~ factor(am) + factor(cyl) + mwt
Model 2: mpg ~ factor(am) + factor(cyl) + mwt + hp
Model 3: mpg ~ factor(am) + factor(cyl) + mwt + hp + drat
Model 4: mpg ~ factor(am) + factor(cyl) + mwt + hp + drat + factor(vs)
  Res.Df RSS Df Sum of Sq    F Pr(>F)  
1     27 183                           
2     26 151  1      31.9 5.35   0.03 *
3     25 151  1       0.2 0.04   0.85  
4     24 143  1       7.5 1.25   0.27  
---
Signif. codes:  0 '***' 0.001 '**' 0.01 '*' 0.05 '.' 0.1 ' ' 1
\end{verbatim}

\subsubsection{Diagnostics for the final
model}\label{diagnostics-for-the-final-model}

\includegraphics{./regCP_files/figure-latex/unnamed-chunk-10.pdf}

\end{document}
